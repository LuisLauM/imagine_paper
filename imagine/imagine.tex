% !TeX root = RJwrapper.tex
\title{imagine: IMAGing engINE, Tools for Application of Image Filters to Data
Matrices}
\author{by Wencheng Lau-Medrano}

\maketitle

\abstract{%
El campo del procesamiento de imágenes abarca un amplio rango de
operaciones que permiten el análisis, síntesis y modificación de la
información contenida en una imagen, entendida esta como una matriz de
datos. Las salidas de dichas operaciones pueden ser otras imágenes
(matrices) o parámetros obtenidos de ellas. Dentro de este campo,
resaltan las operaciones de Convolución y median-filters. En el presente
trabajo, se hace una introducción al paquete imagine, cuyas funciones
permiten la ejecución sobre matrices de datos (imágenes) de las dos
principales operaciones antes mencionadas. A modo de ejemplo, se
mostrarán aplicaciones sobre 1)ecogramas, aplicadas mediante el paquete
oXim (Lau-Medrano, 2016), y 2) detección de frentes de TSM en la costa
de Perú.
}

\section{Introduction}\label{introduction}

El procesamiento de imágenes involucra la aplicación de operaciones del
procesamiento de señales sobre imágenes. Una imagen puede ser definida
como una función bidimensional donde x y y son coordenadas y la amplitud
de f es llamada intensidad o nivel de grises de la imagen en cualquier
punto (Gonzalez \& Woods, 2008). De este modo, una imagen puede ser
entendida como una matriz de valores numéricos en los que cada color y
su intensidad en cada pixel es una representación de los valores en cada
celda de la matriz. A partir de este enfoque numérico, es posible
comprender la aplicación de operaciones numéricas sobre imágenes a fin
de hallar analizar, sintetizar, hallar patrones, calcular parámetros o
modificar la naturaleza de los valores contenidos en una imagen.

Las aplicaciones potenciales provenientes de este enfoque son muchas,
debido a que

Dentro de las principales operaciones en el procesamiento de imágenes,
destacan las de convolución y median-filter.

As Gonzalez \& Woods (XXXX) says ``an image may be defined as a
two-dimensional function f(x, y), where x and y are spatial (plane)
coordinates and the amplitude of f at any pair of coordinates (x, y) is
called the intensity or gray level of the image at that point''.

\address{%
Wencheng Lau-Medrano\\
Instituto del Mar del Perú\\
Esquina Gamarra y General Valle s/n, Chucuito, Callao-Perú\\ +51 (1)208-8650\\
}
\href{mailto:llau@imarpe.gob.pe}{\nolinkurl{llau@imarpe.gob.pe}}

